% Options for packages loaded elsewhere
\PassOptionsToPackage{unicode}{hyperref}
\PassOptionsToPackage{hyphens}{url}
\PassOptionsToPackage{dvipsnames,svgnames,x11names}{xcolor}
%
\documentclass[
  16pt,
  letterpaper,
]{scrbook}

\usepackage{amsmath,amssymb}
\usepackage{lmodern}
\usepackage{setspace}
\usepackage{iftex}
\ifPDFTeX
  \usepackage[T1]{fontenc}
  \usepackage[utf8]{inputenc}
  \usepackage{textcomp} % provide euro and other symbols
\else % if luatex or xetex
  \usepackage{unicode-math}
  \defaultfontfeatures{Scale=MatchLowercase}
  \defaultfontfeatures[\rmfamily]{Ligatures=TeX,Scale=1}
  \setmainfont[]{Roboto}
\fi
% Use upquote if available, for straight quotes in verbatim environments
\IfFileExists{upquote.sty}{\usepackage{upquote}}{}
\IfFileExists{microtype.sty}{% use microtype if available
  \usepackage[]{microtype}
  \UseMicrotypeSet[protrusion]{basicmath} % disable protrusion for tt fonts
}{}
\makeatletter
\@ifundefined{KOMAClassName}{% if non-KOMA class
  \IfFileExists{parskip.sty}{%
    \usepackage{parskip}
  }{% else
    \setlength{\parindent}{0pt}
    \setlength{\parskip}{6pt plus 2pt minus 1pt}}
}{% if KOMA class
  \KOMAoptions{parskip=half}}
\makeatother
\usepackage{xcolor}
\usepackage[margin=0.90in,top = 1.25in,bottom= 1.25in]{geometry}
\setlength{\emergencystretch}{3em} % prevent overfull lines
\setcounter{secnumdepth}{5}
% Make \paragraph and \subparagraph free-standing
\ifx\paragraph\undefined\else
  \let\oldparagraph\paragraph
  \renewcommand{\paragraph}[1]{\oldparagraph{#1}\mbox{}}
\fi
\ifx\subparagraph\undefined\else
  \let\oldsubparagraph\subparagraph
  \renewcommand{\subparagraph}[1]{\oldsubparagraph{#1}\mbox{}}
\fi


\providecommand{\tightlist}{%
  \setlength{\itemsep}{0pt}\setlength{\parskip}{0pt}}\usepackage{longtable,booktabs,array}
\usepackage{calc} % for calculating minipage widths
% Correct order of tables after \paragraph or \subparagraph
\usepackage{etoolbox}
\makeatletter
\patchcmd\longtable{\par}{\if@noskipsec\mbox{}\fi\par}{}{}
\makeatother
% Allow footnotes in longtable head/foot
\IfFileExists{footnotehyper.sty}{\usepackage{footnotehyper}}{\usepackage{footnote}}
\makesavenoteenv{longtable}
\usepackage{graphicx}
\makeatletter
\def\maxwidth{\ifdim\Gin@nat@width>\linewidth\linewidth\else\Gin@nat@width\fi}
\def\maxheight{\ifdim\Gin@nat@height>\textheight\textheight\else\Gin@nat@height\fi}
\makeatother
% Scale images if necessary, so that they will not overflow the page
% margins by default, and it is still possible to overwrite the defaults
% using explicit options in \includegraphics[width, height, ...]{}
\setkeys{Gin}{width=\maxwidth,height=\maxheight,keepaspectratio}
% Set default figure placement to htbp
\makeatletter
\def\fps@figure{htbp}
\makeatother
\newlength{\cslhangindent}
\setlength{\cslhangindent}{1.5em}
\newlength{\csllabelwidth}
\setlength{\csllabelwidth}{3em}
\newlength{\cslentryspacingunit} % times entry-spacing
\setlength{\cslentryspacingunit}{\parskip}
\newenvironment{CSLReferences}[2] % #1 hanging-ident, #2 entry spacing
 {% don't indent paragraphs
  \setlength{\parindent}{0pt}
  % turn on hanging indent if param 1 is 1
  \ifodd #1
  \let\oldpar\par
  \def\par{\hangindent=\cslhangindent\oldpar}
  \fi
  % set entry spacing
  \setlength{\parskip}{#2\cslentryspacingunit}
 }%
 {}
\usepackage{calc}
\newcommand{\CSLBlock}[1]{#1\hfill\break}
\newcommand{\CSLLeftMargin}[1]{\parbox[t]{\csllabelwidth}{#1}}
\newcommand{\CSLRightInline}[1]{\parbox[t]{\linewidth - \csllabelwidth}{#1}\break}
\newcommand{\CSLIndent}[1]{\hspace{\cslhangindent}#1}

\usepackage{fontawesome5}
\makeatletter
\@ifpackageloaded{tcolorbox}{}{\usepackage[many]{tcolorbox}}
\@ifpackageloaded{fontawesome5}{}{\usepackage{fontawesome5}}
\definecolor{quarto-callout-color}{HTML}{909090}
\definecolor{quarto-callout-note-color}{HTML}{0758E5}
\definecolor{quarto-callout-important-color}{HTML}{CC1914}
\definecolor{quarto-callout-warning-color}{HTML}{EB9113}
\definecolor{quarto-callout-tip-color}{HTML}{00A047}
\definecolor{quarto-callout-caution-color}{HTML}{FC5300}
\definecolor{quarto-callout-color-frame}{HTML}{acacac}
\definecolor{quarto-callout-note-color-frame}{HTML}{4582ec}
\definecolor{quarto-callout-important-color-frame}{HTML}{d9534f}
\definecolor{quarto-callout-warning-color-frame}{HTML}{f0ad4e}
\definecolor{quarto-callout-tip-color-frame}{HTML}{02b875}
\definecolor{quarto-callout-caution-color-frame}{HTML}{fd7e14}
\makeatother
\makeatletter
\makeatother
\makeatletter
\@ifpackageloaded{bookmark}{}{\usepackage{bookmark}}
\makeatother
\makeatletter
\@ifpackageloaded{caption}{}{\usepackage{caption}}
\AtBeginDocument{%
\ifdefined\contentsname
  \renewcommand*\contentsname{Table of contents}
\else
  \newcommand\contentsname{Table of contents}
\fi
\ifdefined\listfigurename
  \renewcommand*\listfigurename{List of Figures}
\else
  \newcommand\listfigurename{List of Figures}
\fi
\ifdefined\listtablename
  \renewcommand*\listtablename{List of Tables}
\else
  \newcommand\listtablename{List of Tables}
\fi
\ifdefined\figurename
  \renewcommand*\figurename{Figure}
\else
  \newcommand\figurename{Figure}
\fi
\ifdefined\tablename
  \renewcommand*\tablename{Table}
\else
  \newcommand\tablename{Table}
\fi
}
\@ifpackageloaded{float}{}{\usepackage{float}}
\floatstyle{ruled}
\@ifundefined{c@chapter}{\newfloat{codelisting}{h}{lop}}{\newfloat{codelisting}{h}{lop}[chapter]}
\floatname{codelisting}{Listing}
\newcommand*\listoflistings{\listof{codelisting}{List of Listings}}
\makeatother
\makeatletter
\@ifpackageloaded{caption}{}{\usepackage{caption}}
\@ifpackageloaded{subcaption}{}{\usepackage{subcaption}}
\makeatother
\makeatletter
\@ifpackageloaded{tcolorbox}{}{\usepackage[many]{tcolorbox}}
\makeatother
\makeatletter
\@ifundefined{shadecolor}{\definecolor{shadecolor}{rgb}{.97, .97, .97}}
\makeatother
\makeatletter
\makeatother
\ifLuaTeX
  \usepackage{selnolig}  % disable illegal ligatures
\fi
\IfFileExists{bookmark.sty}{\usepackage{bookmark}}{\usepackage{hyperref}}
\IfFileExists{xurl.sty}{\usepackage{xurl}}{} % add URL line breaks if available
\urlstyle{same} % disable monospaced font for URLs
\hypersetup{
  pdftitle={Introduction to Medical Statistics with Jamovi},
  pdfauthor={Konstantinos I. Bougioukas, PhD},
  colorlinks=true,
  linkcolor={blue},
  filecolor={red},
  citecolor={green},
  urlcolor={blue},
  pdfcreator={LaTeX via pandoc}}

\title{Introduction to Medical Statistics with Jamovi}
\usepackage{etoolbox}
\makeatletter
\providecommand{\subtitle}[1]{% add subtitle to \maketitle
  \apptocmd{\@title}{\par {\large #1 \par}}{}{}
}
\makeatother
\subtitle{1st Edition}
\author{Konstantinos I. Bougioukas, PhD}
\date{August 2, 2022}

\begin{document}
\frontmatter
\maketitle
\ifdefined\Shaded\renewenvironment{Shaded}{\begin{tcolorbox}[interior hidden, frame hidden, boxrule=0pt, borderline west={3pt}{0pt}{shadecolor}, breakable, sharp corners, enhanced]}{\end{tcolorbox}}\fi

\renewcommand*\contentsname{Table of contents}
{
\hypersetup{linkcolor=gray}
\setcounter{tocdepth}{1}
\tableofcontents
}
\listoffigures
\listoftables
\setstretch{1.4}
\mainmatter
\bookmarksetup{startatroot}

\hypertarget{preface}{%
\chapter*{Preface}\label{preface}}
\addcontentsline{toc}{chapter}{Preface}

This textbook is for medical students, doctors, medical researchers,
nurses, members of professions allied to medicine, and all others
concerned with medical data.

While statistics books focus on mathematics, this textbook focuses on
using a computer to conduct data analysis. That means using a
statistical software program, in this case the
\href{https://www.jamovi.org/}{Jamovi} software for statistics and
graphics. Our aim is to keep a balance between mathematical rigor and
readability as well as learning Jamovi and statistics simultaneously.

Most of the examples discussed in this textbook are based on scientific
studies whose data are publicly available. For each example, we provide
the step-by-step application in Jamovi. Readers are encouraged to follow
these steps while reading the textbook so that they can learn statistics
through hands-on experience.

All sections of this textbook are reproducible as they were made using
\href{https://quarto.org/}{Quarto}\textsuperscript{®} which is an
open-source scientific and technical publishing system built on
\href{https://pandoc.org/}{Pandoc}.

To learn more about Quarto books visit
\url{https://quarto.org/docs/books}.

\hypertarget{license}{%
\section*{License}\label{license}}
\addcontentsline{toc}{section}{License}

This textbook is \textbf{free to use}, and is licensed under the
\href{https://creativecommons.org/licenses/by-nc-nd/4.0/}{Creative
Commons Attribution-NonCommercial-NoDerivs 4.0} License.

\bookmarksetup{startatroot}

\hypertarget{introduction}{%
\chapter{Introduction}\label{introduction}}

\hypertarget{statistics-and-medicine}{%
\section{Statistics and Medicine}\label{statistics-and-medicine}}

Although some healthcare professionals may not carry out medical
research, they will definitely be consumers of medical research. Thus,
it is incumbent on them to be able to discern high quality research
studies from low quality, to be able to verify whether the conclusions
of a study are valid and to understand the limitations in methods of a
study. The current emphasis on evidence-based medicine (EBM) requires
that healthcare professionals consider critically all evidence about
whether a specific treatment works and this requires basic statistical
knowledge.

Statistics is not only a discipline in its own right but it is also a
fundamental tool for investigation in all biological and medical
sciences. As such, any serious investigator in these fields must have a
grasp of the basic principles. With modern computer facilities there is
little need for familiarity with the technical details of statistical
calculations. However, a healthcare professional should understand when
such calculations are valid, when they are not and how they should be
interpreted.

The use of statistical methods pervades the medical literature. In a
survey of 350 original articles published in three UK journals of
general practice: \emph{British Medical Journal (General Practice
Section)}, \emph{British Journal of General Practice} and \emph{Family
Practice}, over a one-year period, Rigby et al.~(2004) found that 66\%
used some form of statistical analysis. Another review by Strasak et
al.~(2007) of 91 original research articles published in \emph{The New
England Journal of Medicine} (one of the prestigious peer-reviewed
medical journals) found an even higher percentage (95\%) of using
inferential statistics, for example, hypothesis testing and deriving
estimates. It appears, therefore, that the majority of papers published
in these journals require some statistical knowledge for a complete
understanding.

To students schooled in the `hard' sciences of physics and chemistry it
may be difficult to appreciate the variability of biological data. If
one repeatedly puts blue litmus paper into acid solutions it turns red
100\% of the time, not most (say 95\%) of the time. In contrast, if one
gives aspirin to a group of people with headaches, not all of them will
experience relief. Penicillin was perhaps one of the few `miracle' cures
where the results were so dramatic that little evaluation was required.
Absolute certainty in medicine is rare.

Measurements on human subjects seldom give exactly the same results from
one occasion to the next. For example, O' Sullivan et al (1999), found
that systolic blood pressure (SBP) in normal healthy children has a wide
range, with 95\% of children having SBPs below 130 mmHg when they were
resting, rising to 160 mmHg during the school day, and falling again to
below 130 mmHg at night. Furthermore, Hansen et al.~(2010) in a study of
over 8000 subjects found that increasing variability in blood pressure
over 24 hours was a significant and independent predictor of mortality
and a cardiovascular and stroke events.

This variability is also inherent in responses to biological hazards.
Most people now accept that cigarette smoking causes lung cancer and
heart disease, and yet nearly everyone can point to an apparently
healthy 80-year-old who has smoked for many years without apparent ill
effect. Although it is now known from the report of Doll et al (2004)
that about half of all persistent cigarette smokers are killed by their
habit, it is usually forgotten that until the 1950s, the cause of the
rise in lung cancer deaths was a mystery and commonly associated with
general atmospheric pollution such as the exhaust fumes of cars. It was
not until the carefully designed and statistically analysed
case--control and cohort studies of Richard Doll and Austin Bradford
Hill and others, that smoking was identified as the true cause. Enstrom
et al.~(2003) moved the debate on to ask whether or not passive smoking
causes lung cancer. This is a more difficult question to answer since
the association is weaker. However, studies by Cao et al.~(2015) have
now shown that it is a major health problem and scientists at the
International Agency for Rsearch on Cancer (IARC) have concluded that
there is sufficient evidence that second-hand smoke causes lung cancer
(IARC 2012). Restrictions on smoking in public places have been imposed
to smokers.

With such variability, it follows that in any comparison made in a
medical context, such as people on different treatments, differences are
almost bound to occur. These differences may be due to real effects,
random variation or variation in some other factor that may affect an
outcome. It is the job of the analyst to decide how much variation
should be ascribed to chance or other factors, so that any remaining
variation can be assumed to be due to a real effect. This is the art of
statistics.

\hypertarget{why-jamovi}{%
\section{Why Jamovi?}\label{why-jamovi}}

Jamovi is a new fee open ``3rd generation'' statistical software that is
built on top of the programming language R, which has become extremely
popular among data scientists during the last two decades. Designed from
the ground up to be easy to use, Jamovi is a compelling alternative to
costly statistical products such as SPSS and SAS.

Some other advantages are:

\begin{enumerate}
\def\labelenumi{\arabic{enumi}.}
\item
  A user guide and community resources from the jamovi website
\item
  Integration with R
\item
  It provides informative tables and neat visuals.
\end{enumerate}

\bookmarksetup{startatroot}

\hypertarget{sampling-methods-and-study-designs}{%
\chapter{Sampling methods and study
designs}\label{sampling-methods-and-study-designs}}

\bookmarksetup{startatroot}

\hypertarget{probability-and-distributions}{%
\chapter{Probability and
distributions}\label{probability-and-distributions}}

\bookmarksetup{startatroot}

\hypertarget{normal-distribution}{%
\chapter{Normal distribution}\label{normal-distribution}}

\bookmarksetup{startatroot}

\hypertarget{foundations-for-statistical-inference}{%
\chapter{Foundations for statistical
inference}\label{foundations-for-statistical-inference}}

\bookmarksetup{startatroot}

\hypertarget{inference-for-numerical-data-2-samples}{%
\chapter{Inference for numerical data: 2
samples}\label{inference-for-numerical-data-2-samples}}

Two sample t-test (Student's t-test) can be used if we have two
independent (unrelated) groups (eg., males-females, unmatched
case-controls, treatment-non treatment) and one quantitative variable of
interest (e.g., age, weight, systolic blood pressure). For example, we
may want to compare the age in males and females or the weights in two
groups of children, each child being randomly allocated to receive
either a dietary supplement or placebo.

\begin{tcolorbox}[enhanced jigsaw, opacitybacktitle=0.6, bottomrule=.15mm, titlerule=0mm, colback=white, leftrule=.75mm, breakable, colframe=quarto-callout-warning-color-frame, bottomtitle=1mm, colbacktitle=quarto-callout-warning-color!10!white, toprule=.15mm, left=2mm, opacityback=0, coltitle=black, arc=.35mm, title={\textbf{Assumptions for conducting a Student's t-test}}, rightrule=.15mm, toptitle=1mm]

\begin{enumerate}
\def\labelenumi{\arabic{enumi}.}
\tightlist
\item
  The groups are independent
\item
  The outcome of interest is continuous
\item
  The data is normally distributed in both groups
\item
  The data in both groups have similar standard deviations
\end{enumerate}

\end{tcolorbox}

\[t = \frac{\bar{x}_{1} - \bar{x}_{2}}{se_{dif}}\]

\bookmarksetup{startatroot}

\hypertarget{inference-for-numerical-data-2-samples-1}{%
\chapter{Inference for numerical data: \textgreater2
samples}\label{inference-for-numerical-data-2-samples-1}}

\bookmarksetup{startatroot}

\hypertarget{inference-for-categorical-data}{%
\chapter{Inference for categorical
data}\label{inference-for-categorical-data}}

\bookmarksetup{startatroot}

\hypertarget{correlation}{%
\chapter{Correlation}\label{correlation}}

\bookmarksetup{startatroot}

\hypertarget{simple-linear-regression}{%
\chapter{Simple linear regression}\label{simple-linear-regression}}

\bookmarksetup{startatroot}

\hypertarget{reporting-the-results-of-statistical-analysis}{%
\chapter{Reporting the results of statistical
analysis}\label{reporting-the-results-of-statistical-analysis}}

\bookmarksetup{startatroot}

\hypertarget{references}{%
\chapter*{References}\label{references}}
\addcontentsline{toc}{chapter}{References}

\hypertarget{refs}{}
\begin{CSLReferences}{0}{0}
\end{CSLReferences}


\backmatter

\end{document}
